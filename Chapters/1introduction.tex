\section{Introduction} \label{chap:introduction}
The desirability of minimum wage policy has been a heavily debated subject for over a century. On one hand, setting a wage floor can ensure that those who work earn enough to live decently, if not comfortably \citep{Macrosty1898}. On the other, fixing the price of labour might unbalance supply and demand, leading to increased unemployment and the poverty risks that entails. There is still no consensus on the existence or size of this disemployment effect despite hundreds of papers already published on the subject, mainly based on the state-level variation in the US.

The introduction of a national minimum wage in Germany (Jan 1st 2015, €8.50/hr) is a welcome opportunity to obtain a fresh angle on the issue. As a major industrial economy, Germany is more comparable to other western economies than say, Indonesia \citep{Pratomo2016}. The minimum wage introduced is also considerable, exceeding hourly wages in the restaurant and fastfood sector for over 50\% of its employees.\footnote{Own calculations based on the German Socio-Economic Panel.} At 48\% of the median wage, it is similar to long established minimum wages in neighbouring countries.\footnote{Source: OECD} Unlike US studies, we are not limited to geography-level data, but can instead analyse firms directly thanks to our access to the nationally representative, firm level, database of Germany's largest credit rating agency.

The national character of the new minimum wage means the panel approaches standard in the literature cannot be applied (e.g. \citet{Neumark1992, Allegretto2013, Dube2010}). Instead, we exploit differences in pre-treatment wage levels across sectors to identify the effect minimum wages have on employment and turnover levels. We link firms in heavily affected sectors to those in \emph{de facto} unaffected sectors, matching on past employment paths and forward looking creditratings. We surround the matching period by two testing periods (2010-2011 and 2013-2014) to evaluate our matching process.

Our results indicate employment barely responded to the minimum wage introduction. Even among micro firms in East Germany, we only found a reduction in employment growth of 1.8 percentage points. Overall, employment growth in the East of treated firms was 0.8 pp lower than in the control group, in the West the point estimate is indistinguishable from zero. In headcounts, this equates to just 21 482 jobs lost, or 0.05\% of total employment.

Our results are largely in line with other evaluation studies. For example, \citet{Caliendo2018le}, exploiting regional differences in the bite of the minimum wages, find a reduction in overall employment of 0.5\%, concentrated amongst mini-jobbers.\footnote{The mini-job statute in Germany significantly reduces social security contributions for (very) low earning employees.} \citet{Garloff2016} takes this approach one step further by creating region-age-gender cells, each with specific minimum wage bites and employment evolutions. They also finds no meaningful overall employment effect, only a shift from mini jobs to regular employment. Turning to survey data, \citet{Bossler2016} find that employment grew by 1.9\% less in firms which reported employing employees earning less than €8.50/hr in the the IAB Establishment Panel. Relative to total employment that represents a reduction of about 0.15\%.

This study ties in to the wider international debate on the welfare effects of minimum wages and discussion on which mechanisms could explain the lack of employment effects found in some studies (e.g. \citet{Cengiz2018, Allegretto2013, Dube2016a}) but not in others (e.g. \citet{Neumark1992, Liu2016, Neumark2014a}). Our estimates suggest that a part of the cost shock is absorbed by higher turnover, either through firms raising prices or wealthier consumers leading to more product demand. This corroborates existing studies in the US (\citet{Aaronson2001, Allegretto2018}) and Hungary (\citet{Cengiz2018}). Credit ratings remain largely unaffected which indicates firm profit was not meaningfully impacted, this contrasts with \citet{Draca2011}, who find that the national minimum wage introduced in the UK in 1999 lowered profitability of affected firms by 2.7\%. %Expand further on this?  