\section{Related Literature} \label{chap:literature}
% Overview + Neumark & Wascher
There are essentially three strands in the international minimum wage literature. There are the US-based studies which find strong negative employment effects from the minimum wage, there are US-based studies which find no such effects and then there is literature about the rest of the world. The first, which sees minimum wages as potentially very harmful in developed economies is centered around the \citet{Neumark1992} study. They compare US states over time using (mainly) standard twoway fixed effects models and find that a 10\% increase in the minimum wage lead to a reduction in teen employment of 2-3\%, which is in line with findings from earlier time series models \citep{Brown82}. 

% Card & Krueger
The second strand does not find any disemployment effects and tries to uncover how firms cope with the increases in cost due to the minimum wage. The most notable example is \citet{Card1994}, who conducted surveys in fastfood restaurants in Pennsylvania and New Jersey before and after the NJ minimum wage was increased. Contrary to expectations, they found an \emph{increase} in employment in NJ relative to PA. Additionally, they note that prices in NJ increased more, suggesting restaurants were able to pass through their cost increase to customers.

% DLR & ADR
More recently, \citet{Dube2010} generalised this approach by comparing all counties across state border with a minimum wage differential. They find no statistically significant employment effect in either direction although their standard errors are quite large.\footnote{That is, the mean estimate is 0.016, but at 95\% confidence, they cannot rule out employment elasticities down to -0.178.} \citet{Allegretto2011} come to the same conclusion when they add state-specific time trends to traditional twoway fixed effects specification, or when they restrict the analysis to comparisons within census divisions. 

% Channels of adjustment (prices)
Prices also remain a likely pressure relief valve. \citet{Aaronson2001} use official Bureau of Labor Statistics panel data to show that restaurants respond to minimum wage hikes by increasing their prices one-to-one relative to costs. \citet{Allegretto2018} instead analyse a single high-impact event and find that restaurants in San Jose, CA increased their prices similarly after the 25\% hike in the minimum wage. Other channels of adjustment include pay compression \citep{Hirsch2015} and cost savings through reduced employee turnover \citep{Dube2016a}. 

% International literature
These two strands frequently clash, to the extent that e.g. the ILR Review occasionally asks both sides to submit comments (most recently, \citet{Neumark2017} and \citet{Allegretto2017}). The non-US literature is tied less strongly to the overall debate. \citet{Neumark2006} dutifully provides an overview of minimum wage studies across the world, but quite tellingly, none of them feature in the two ILR Review papers, which reflect the current state of the art. There are many studies based on non-western economies (e.g. Indonesia [\citet{Comola2011}, \citet{Yamada2016}]; Brazil [\citet{Lemos2009}]; China [\citet{Long2016}]; Mexico [\citet{Feliciano1998}]), which are meaningful to the type of country analysed, but are only of limited use to policymakers in western nations. The only two countries which currently feature in the US-based debate are Canada (e.g. tbd) and the United Kingdom (e.g. \citet{Machin2003}, \citet{Metcalf2008}), as they have the right combination of similarity to the US, data availability and particularly variation in the minimum wage.

% German literature
Since its introduction of a national minimum wage on January 1st 2015, Germany should be added to this select list. It is the fourth largest economy in the world, with a high standard of publicly or privately available datasets and a meaningful minimum wage shock, affecting one in five employees nationwide (and up to 65\% in certain sectors). 