\section{Conclusion} \label{chap:conclusion}
In this paper we present a novel approach to analyse the employment effects of the national German minimum wage introduced on January 1st 2015 and discuss how this policy shock can inform the wider international minimum wage debate, which is particularly active in the US. We start from individual hourly wages to determine which sectors were vulnerable to the new wage floor and which should remain unaffected.\footnote{We exclude the grey zone in between from our analyses.} Then we turn to firm level data and match firms in treated sectors to similar firms in unaffected sectors. The richness of the (proprietary) dataset used allows us to not only match on past employment, but also on the firm's credit score in 2014, which represents expectations about its future.

We find a very small overall effect, suggesting employment in treated firms grew 1.1\% slower than in their untreated counterparts, equivalent to 54 000 jobs lost. Although this result is in line with existing studies \citep{Bossler2016, Caliendo2018le}, we also show that it is rather fragile. For example, changing the thresholds that assign treatment status or adding Italian firms as potential controls already leads to respectively insignificant and minor positive employment effect estimates.

The same cannot be said for our second main result, that the restaurant sector benefited strongly from increased bottom decile earnings across Germany. This finding matters particularly because in the main US-based studies, it is either restaurant or teenage employment that is investigated. If similar mechanisms are at play in the US as in Germany, it would be misleading to extrapolate the restaurant-based employment effects to the rest of the US economy, which may enjoy the same pricing flexibility and product demand effects.

Overall, we can conclude that the catastrophic labour market effects predicted by ex ante studies (over one million job losses) have failed to materialise. Instead, it led to robust wage growth \citep{Bossler2016} and at most very limited employment losses, and that only in particular sectors. In others, most notably the food services sector, the national minimum wage even led to employment growth, suggesting there is a place for minimum wage policy in a social planner's toolbox.