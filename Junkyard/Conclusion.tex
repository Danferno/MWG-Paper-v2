
\section{Conclusion}
In this paper we show that the many different approaches followed in the literature to identify the employment elasticity of the minimum wage can be nested within the multifactor error structure framework. However, as they impose fairly stong functional form restrictions, it is unclear how robust their results are. Instead, we turn to the two workhorses of the cross sectional dependence literature - the Common Correlated Effects (CCE) and the Interactive Fixed Effect (IFE) models. Unlike the traditional models, which combat spatial heterogeneity by limiting the analysis to local variation or imposing linear trends, these two frameworks require large panels rather than strong assumptions. We find no evidence for disemployment effects. 

In addition, our statistical tests indicate that there is a high risk of misspecification in these static models, a claim frequently made in the literature. Luckily, the CCE model is readily extended to a dynamic panel setting. The overall conclusions remain the same.

The only specifications that lead to significant negative employment effects are the pooled OLS and two-way fixed effects models. There is, to our knowledge, no reason to expect these models to be more credible than the richer frameworks controlling for spatial heterogeneity or dynamic effects, which have shown to be consistent even if there is no spatial heterogeneity present in the data generating process \citep{Eberhardt2010}. On the contrary, the robustness checks executed in Section \ref{chap:robustness} show that the standard models even find significant disemployment effects in sectors unaffected by minimum wage legislation, whereas the richer models still produce elasticities around zero.
%One caveat is that controlling for cross-sectional dependence ought only to considerably change parameter estimates (as is the case here) if the error term and the independent variable are correlated. In other words, the large changes brought about by controlling for spatial heterogeneity indicate that perhaps there is an endogeneity problem after all.